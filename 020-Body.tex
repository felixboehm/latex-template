\chapter{Latex}

\section{Tools}

MiKTeX: \url{https://miktex.org/download}
TeXLive: \url{http://tug.org/texlive/}
 (or alternative LaTeX-systems).
 
 A good editor is essential. Sometimes combined editors and compilers (e.g. TeXShop) can be really productive. Make sure you learn the use of synchronized navigation then.

In modern software development, we often work with \ac{API}s, use \ac{CI} systems, and deploy applications through \ac{CLI} tools.

A vector graphic is one where strokes remain strokes even at the highest resolutions: e.g. the Figure~\ref{fig:spiral} or the Logo on the \hyperref[titlePage]{Titelblatt} (notice: you can click from here to there).
Many tools generate vector-graphics for plots from any data-set. E.g. Plotly (with the use of the Browser-Print), MatPlotLib or even OpenOffice, LibreOffice or MS-Excel.

\section{Literature References}
Here is an example of a reference with a page-number: \textcite[p.~6]{DueckKo:2016}


\section{Pictures}

\begin{figure}[h]
\centering
\includegraphics[width=8cm]{pics/spiral.pdf}
\caption{A spiral... smooth vector-based with a clean parametrisation! \\ Nothing to do with \textcite{Gage:18}}\label{fig:spiral}
\end{figure}
\FloatBarrier

\section{Tables}

\begin{table}[H]
\small
\centering
\begin{tabular}{p{5cm}|l|p{3cm}}
`` Industrial era '' &  ``Jobs '' & `` Wanted: Upgrade''' \\ \hline
Parts exchanger & Fitter & mechatronics specialist \\
eShop & reseller & `` Client-suggester'' \\
`` Coding-guru''' & Softwaredesign & Whole-life designer \\
JA! Gut \& Günstig & brand-names & `` Life-Style Feeling'' \\
Internetbanking & Bank clerk & Customer adviser \\
Robots & Specialist & Machine supervisor \\
Bush & Gardener & Nature-sculptor \\
Painting & Painter & Interior Design \\
 &  & \\
\end{tabular}
\caption[Downgrade and upgrade of job denominations]{Downgrade and Upgrade of job denominations \\ \ \ \ \textcite{DueckKo:2016}}
\label{tab:Downgrade and Upgrade of job denominations}
\end{table} 

\section{Listes}

\begin{itemize}
 \itemsep0pt
 \item one
 \item twoi
 \item threei
\end{itemize}

\begin{enumerate}
 \itemsep0pt
 \item first
 \item second
 \item third
\end{enumerate}


\section{Formulæ}

A formula can be written inline, e.g. as $ \frac{d}{dx}\mbox{arctg}(x) = \frac{1}{1+x^2}$ or, in centered math:

\begin{equation}  \frac{d}{dx}\mbox{arctg}(x) = \frac{1}{1+x^2} \label{arctanderivative}\end{equation}

Notice that formulæ that are centered start bigger (technically, they start in \verb+\displaystyle+) than they start inline (technically, they start in \verb+\textstyle+ all subsequents reductions, e.g. an exponent, goes to \verb+\scriptstyle+ then \verb+\scriptscriptstyle+). Indeed a best effort is made so that inline formulæ do not change the line height which would bother the eye of a reader.

Formulæ can be given a number and a label. Numbering happens automatically with \verb+\begin{equation}+ and \verb+\end{equation}+ and can be avoided if enclosing the formula between \verb+\[+ and \verb+\]+. If using the \verb+\label+ macro inside, you can refer automatically to this equation using \verb+\ref{label}+. E.g. Thanks to equation~\ref{arctanderivative} one dare say that:

\begin{equation} \int_0^t \frac{1}{1+x^2} dx = \mbox{arctan}(t) \end{equation}


\section{Tools and Code}

Many users of this template will want to include some code.

The simplest way to do so is to use the \verb+\verb+ macro which is followed by a sign, then some code, then the sign again to close. This is the inline version which works as in: 


\begin{verbatim}
As we could calculate with \textcite{Wolfram_Alpha} using 
\verb_integrate 1 / pi e ^ (t/pi) from zero to infinity_.
\end{verbatim}

which yields: 

As we could calculate with \textcite{Wolfram_Alpha} using \verb_integrate 1 / pi e ^ (t/pi)_ from zero to infinity.


The multiline version of this is called \verb+\begin{verbatim}+ and finishes with \verb+\end{verbatim}+.


\section{Citation examples}

Monography \parencite[p.~22]{con:infra} 

Collection \parencite{sammelband} 

Article \parencite{article1}
